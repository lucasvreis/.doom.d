% Created 2021-08-04 qua 21:26
% Intended LaTeX compiler: pdflatex
\documentclass[11pt]{article}
\usepackage[utf8]{inputenc}
\usepackage[T1]{fontenc}
\usepackage{graphicx}
\usepackage{grffile}
\usepackage{longtable}
\usepackage{wrapfig}
\usepackage{rotating}
\usepackage[normalem]{ulem}
\usepackage{amsmath}
\usepackage{textcomp}
\usepackage{amssymb}
\usepackage{capt-of}
\usepackage{hyperref}
\usepackage{tikz}
\usepackage{tikz-cd}
\author{Lucas Reis}
\date{\today}
\title{Minha Configuração do Emacs}
\hypersetup{
 pdfauthor={Lucas Reis},
 pdftitle={Minha Configuração do Emacs},
 pdfkeywords={},
 pdfsubject={},
 pdfcreator={Emacs 28.0.50 (Org mode 9.5)}, 
 pdflang={English}}
\begin{document}

\maketitle
\tableofcontents

Vamos usar o binding léxico:
\begin{verbatim}
;;; -*- lexical-binding: t; -*-
\end{verbatim}

\section{Doom}
\label{sec:orge9ac282}
\subsection{Zen}
\label{sec:org76d1931}
Ver também \href{lisp/minor/writeroom-mode.el}{writeroom-mode} e \href{lisp/minor/mixed-pitch.el}{mixed-pitch}.
\begin{verbatim}
(setq +zen-text-scale 0)
\end{verbatim}

\section{Estético}
\label{sec:orge173df1}
\subsection{Tema}
\label{sec:orgdb6752c}
\begin{verbatim}
(setq doom-theme 'doom-tomorrow-day)
\end{verbatim}

\subsubsection{Esconder temas}
\label{sec:org5cf09e5}

Deixa a seleção menos distrativa, removendo os temas não-doom e os temas da
lista abaixo.

\begin{verbatim}
(defvar my/custom-themes-exclude
  '(doom-acario-light
    doom-acario-dark
    doom-badger
    doom-dark+
    doom-henna
    doom-homage-white
    doom-Iosvkem
    doom-1337
    doom-molokai
    doom-sourcerer
    doom-peacock
    doom-wilmersdorf
    doom-manegarm
    doom-ephemeral
    doom-nova
    doom-opera
    doom-zenburn
    doom-ayu-mirage
    doom-vibrant
    doom-ir-black
    doom-old-hope
    doom-miramare
    doom-monokai-spectrum
    doom-monokai-ristretto))

(advice-add 'custom-available-themes :filter-return
            (lambda (l)
              (seq-difference
               (seq-filter (lambda (x) (s-prefix? "doom-" (symbol-name x))) l)
               my/custom-themes-exclude #'eq)))
\end{verbatim}
\subsection{Faces}
\label{sec:org9f19b59}
\subsubsection{Fontes}
\label{sec:orgedaa4e9}

\begin{verbatim}
(setq doom-font                (font-spec :family "Iosevka" :size 19 :weight 'normal)
      doom-variable-pitch-font (font-spec :family "Overpass" :size 19 :weight 'light)
      doom-serif-font          (font-spec :family "IBM Plex Mono" :weight 'light))
      ;; doom-unicode-font        (font-spec :family "JuliaMono" :weight 'normal))

;; Colocamos uma ordem de prioridade para tentar ter todos os unicodes e emojis.
(add-hook! 'after-setting-font-hook
  (set-fontset-font t 'unicode "JuliaMono")
  (set-fontset-font t 'unicode "Twemoji" nil 'append))
\end{verbatim}

\subsubsection{Código (prog)}
\label{sec:orgbc4606a}
Deixamos os comentários itálicas, e os \texttt{keywords} oblíquos.

\begin{verbatim}
(custom-set-faces!
  '(font-lock-comment-face :slant italic)
  '(font-lock-keyword-face :slant oblique))
\end{verbatim}

\subsubsection{@Yasnippet}
\label{sec:org64b28da}
\begin{verbatim}
(custom-set-faces!
  `(yas-field-highlight-face
    :inherit nil
    :background ,(doom-blend "#99f056" (doom-color 'bg) 0.1)))
\end{verbatim}
\subsubsection{@Org}
\label{sec:org42d8e0d}
\begin{verbatim}
(custom-set-faces!
  `(org-latex-and-related :foreground ,(doom-color 'cyan) :weight normal)
  '(font-latex-math-face :inherit org-latex-and-related)
  '(org-block-begin-line :extend t))
(custom-theme-set-faces! 'doom-flatwhite
  `(org-latex-and-related :foreground nil :background ,(doom-color 'fw-green-blend)))
(custom-set-faces!
  '(outline-1 :weight extra-bold :height 1.35)
  '(outline-2 :weight bold :height 1.28)
  '(outline-3 :weight bold :height 1.20)
  '(outline-4 :weight semi-bold :height 1.09)
  '(outline-5 :weight semi-bold)
  '(outline-6 :weight semi-bold)
  '(outline-8 :weight semi-bold)
  '(outline-9 :weight semi-bold))
\end{verbatim}

\subsubsection{Ícones}
\label{sec:orgc9026d2}
Ajusta o tamanho dos ícones.

\begin{verbatim}
(setq all-the-icons-scale-factor 0.88)
\end{verbatim}

\subsection{Modeline}
\label{sec:org8438328}

\begin{verbatim}
;;; tangle/minor/doom-modeline.el -*- lexical-binding: t; -*-

(doom-modeline-def-segment buffer-name
  "Display the current buffer's name, without any other information."
  (concat
    (doom-modeline-spc)
    (doom-modeline--buffer-name)))

(doom-modeline-def-segment pdf-icon
  "PDF icon from all-the-icons."
  (concat
    (doom-modeline-spc)
    (doom-modeline-icon 'octicon "file-pdf" nil nil
                        :face (if (doom-modeline--active)
                                  'all-the-icons-red
                                'mode-line-inactive)
                        :v-adjust 0.02)))

(defun doom-modeline-update-pdf-pages ()
  "Update PDF pages."
  (setq doom-modeline--pdf-pages
        (let ((current-page-str (number-to-string (eval `(pdf-view-current-page))))
              (total-page-str (number-to-string (pdf-cache-number-of-pages))))
          (concat
            (propertize
              (concat (make-string (- (length total-page-str) (length current-page-str)) 32)
                    " P" current-page-str)
              'face 'mode-line)
            (propertize (concat "/" total-page-str) 'face 'doom-modeline-buffer-minor-mode)))))

(doom-modeline-def-segment pdf-pages
  "Display PDF pages."
  (if (doom-modeline--active) doom-modeline--pdf-pages
    (propertize doom-modeline--pdf-pages 'face 'mode-line-inactive)))

(doom-modeline-def-modeline 'pdf
  '(bar window-number pdf-pages pdf-icon buffer-name)
  '(misc-info matches major-mode process vcs))

\end{verbatim}

\begin{verbatim}
(custom-set-faces!
  '(mode-line :height 110 :family "JuliaMono")
  '(mode-line-inactive :height 110 :family "JuliaMono")
  '(doom-modeline-buffer-modified :foreground "#c63")
  '(doom-modeline-info :foreground "white"))
(setq! +modeline-height 26)
\end{verbatim}

\subsection{Divisores de janelas}
\label{sec:org277942a}

\begin{verbatim}
(setq window-divider-default-bottom-width 2   ; default is 1
      window-divider-default-right-width  2)  ; default is 1
\end{verbatim}

\subsection{Dashboard}
\label{sec:orgf42ac29}
\subsubsection{Estético}
\label{sec:orgbfe7fe6}

Pequenas modificações ao dashboard:

\begin{verbatim}
;; Desabilita o "benchmark"
(remove-hook 'window-setup-hook #'doom-display-benchmark-h)
\end{verbatim}

\subsubsection{Banner}
\label{sec:org8a4c71c}

\begin{verbatim}
(setq +doom-dashboard-functions '(doom-dashboard-widget-shortmenu
                                  doom-dashboard-widget-loaded))
\end{verbatim}

A fonte padrão é itálica e deixa os acrobatas tortos, vamos resolver isso:

\begin{verbatim}
(custom-set-faces!
  '(doom-dashboard-banner
    :inherit font-lock-comment-face
    :slant normal))
\end{verbatim}

\subsection{Meu \texttt{zen-mode}}
\label{sec:org80f4d57}

\begin{verbatim}
(defvar my/zen-enabled nil)

(defun my/zen-enable ()
  (interactive)
  (hide-mode-line-mode +1)
  (setq-local my/zen-enabled t
              line-spacing 0.1
              display-line-numbers nil))

(defun my/zen-disable ()
  (interactive)
  (hide-mode-line-mode -1)
  (setq-local my/zen-enabled nil
              line-spacing 0
              display-line-numbers t))

(defun my/zen-toggle ()
  (interactive)
  (if my/zen-enabled
      (my/zen-disable)
    (my/zen-enable)))

(map! :leader "t z" #'my/zen-toggle)
\end{verbatim}

\section{Minha biblioteca}
\label{sec:orgdf5fc86}
\subsection{Pasta-bibioteca}
\label{sec:org96002e9}

Com algumas bibliotecas externas, por exemplo.

\begin{verbatim}
(let ((default-directory "~/.doom.d/lisp/lib"))
  (normal-top-level-add-subdirs-to-load-path))
(add-load-path! "lisp/lib")
\end{verbatim}

\subsection{Funções (não interativas)}
\label{sec:orgae01f86}
\subsubsection{Conselhos}
\label{sec:orgb1a8bfa}
Para desabilitar as mensagens chatas:
\begin{verbatim}
(defun advice--inhibit-message (f &rest r) (let ((inhibit-message t)) (apply f r)))
\end{verbatim}

\subsubsection{Checa se objeto é uma lista de strings}
\label{sec:orga2e5768}
\begin{verbatim}
(defun string-list-p (x) (and (listp x) (--all? (stringp it) x)))
\end{verbatim}

\subsection{Funções (interativas)}
\label{sec:org6418272}
\subsubsection{Excluir todos os conselhos}
\label{sec:orgd53a425}
\begin{verbatim}
(defun advice-unadvice (sym)
  "Remove all advices from symbol SYM."
  (interactive "aFunction symbol: ")
  (advice-mapc (lambda (advice _props) (advice-remove sym advice)) sym))
\end{verbatim}

\section{Emacs em geral}
\label{sec:org9572b80}
\subsection{Scroll}
\label{sec:org4892092}
\begin{verbatim}
(setq mouse-wheel-scroll-amount '(1 ((shift) . 1)) ;; one line at a time
      mouse-wheel-progressive-speed nil ;; don't accelerate scrolling
      confirm-kill-emacs nil
      scroll-step 1) ;; keyboard scroll one line at a time
\end{verbatim}

\subsection{Edição de texto}
\label{sec:org3cc2396}
\subsubsection{Atalhos}
\label{sec:org36a08b4}
\subsubsection{Variáveis}
\label{sec:orgeaff0ac}
\begin{verbatim}
(setq-default fill-column 80)

(setq company-idle-delay 0.01
      company-minimum-prefix-length 4)

(setq mouse-drag-and-drop-region t
      mouse-drag-and-drop-region-cut-when-buffers-differ t
      mouse-drag-and-drop-region-show-tooltip nil)

(setq default-input-method "TeX")
\end{verbatim}

Deixa o \texttt{text-scale-mode} mais devagar.

\begin{verbatim}
(setq text-scale-mode-step 1.05)
\end{verbatim}
\subsubsection{Salvando}
\label{sec:org403bc5c}
Desabilita a mensagem de salvamento.

\begin{verbatim}
(advice-add 'save-buffer :around #'advice--inhibit-message)
\end{verbatim}

\subsubsection{Piscar o cursor\hfill{}\textsc{ARCHIVE}}
\label{sec:org371e6dc}
\subsubsection{Abbrev}
\label{sec:orgb99a92e}
Meu deus, como eu passei tanto tempo (4 meses) sem saber da existência
desta coisa MARAVILHOSA??? É INCRÍVEL!!!!

\begin{verbatim}
(add-hook! 'text-mode-hook
           (abbrev-mode +1))

(setq abbrev-file-name (concat doom-private-dir "abbrev_defs"))
\end{verbatim}

\subsection{Pesquisa \& regex}
\label{sec:orgb6e2528}
\begin{verbatim}
(pcre-mode +1)
\end{verbatim}

\subsection{Menu de contexto}
\label{sec:org8172c01}

Retirado de:
\href{https://somecallmespace.com/emacs-context-menu.html}{some call me Space | Emacs Context Menu}

\begin{verbatim}
(require 'context-menu)
(map! [mouse-3] 'my-context-menu)
\end{verbatim}

\subsection{Popups}
\label{sec:org9f4e53b}
\begin{verbatim}
(setq +popup-defauts
      '(:side bottom
        :height 0.3
        :width 130
        :quit t
        :select ignore
        :ttl 5))

(setq +popup-default-alist
      '((window-height . 0.3)
        (reusable-frames . visible)))

\end{verbatim}

\subsection{Outros\hfill{}\textsc{completion}}
\label{sec:org07835dd}

\begin{verbatim}
(remove-hook! '(org-mode-hook text-mode-hook) #'flyspell-mode)

(setq vterm-shell "zsh"
      delete-by-moving-to-trash t
      mouse-autoselect-window nil)
\end{verbatim}

\subsection{Maximizar a janela?}
\label{sec:org548ce9e}

Maximizar a janela ao iniciar?

\begin{verbatim}
;; (add-to-list 'initial-frame-alist '(fullscreen . maximized))
\end{verbatim}

\section{Configurações de pacotes}
\label{sec:orgaf2d49b}
\subsection{{\bfseries\sffamily TODO} Use-packages (temporário)}
\label{sec:orgfc51f71}

Primeiro, alguns pacotes configurados com o \texttt{use-package}:

\begin{verbatim}
(use-package prettify-utils
  :after (org latex))

(use-package tree-sitter
  :after doom-first-file-hook
  :config
  (require 'tree-sitter-langs)
  (global-tree-sitter-mode)
  (add-hook 'tree-sitter-after-on-hook #'tree-sitter-hl-mode))

(use-package scroll-on-drag
  :bind ([down-mouse-2] . #'scroll-on-drag))

(defun yas-get-snippet (mode key)
       (yas--fetch (yas--get-snippet-tables mode) key))

(use-package laas
  :commands (laas-mode))

(use-package lean4-mode
  :commands (lean4-mode))

(use-package mamimo
  :hook ((org-mode latex-mode markdown-mode) . mamimo-mode))
\end{verbatim}

\subsection{{\bfseries\sffamily TODO} Packages.el (temporário)}
\label{sec:org917055a}

Isso é temporário, enquanto eu não organizo o meu antigo arquivo \texttt{packages.el}.
\begin{verbatim}
;; -*- no-byte-compile: t; -*-
;;; $DOOMDIR/packages.el

;; latex
;; (package! latex-preview-pane)
(package! aas :recipe (:host github :repo "ymarco/auto-activating-snippets"))

;; doom
(package! writegood-mode :disable t)
(package! hl-line :disable t)
(package! smooth-scrolling)
(package! scroll-on-drag)
(package! nyan-mode)

;; Tree-sitter
(package! tree-sitter)
(package! tree-sitter-langs)

(package! lsp-treemacs)

(unpin! lsp-mode)
(unpin! company-box)
(unpin! consult)
(unpin! doom-themes)
(unpin! vertico)
(unpin! treemacs)

(package! benchmark-init :recipe (:host github :repo "kekeimiku/benchmark-init-el"))

(package! company-math)
(package! math-symbol-lists)
(package! company-math)
(package! real-auto-save)
\end{verbatim}

\subsection{Major modes}
\label{sec:org4316566}
\subsubsection{Org}
\label{sec:org9dbdc7d}
\begin{enumerate}
\item Variáveis
\label{sec:org94a49b4}
\begin{verbatim}
(setq org-directory "~/Lucas/org"
      org-latex-packages-alist '(("" "tikz" t) ("" "tikz-cd" t))
      org-support-shift-select t
      org-hide-emphasis-markers t
      org-src-window-setup 'plain
      org-highlight-latex-and-related '(native script)
      org-indent-indentation-per-level 1)
\end{verbatim}

\item Gutter
\label{sec:orgdbbfd77}

O \texttt{git-gutter} não funciona bem com o \texttt{org-indent-mode}:

\begin{verbatim}
(push 'org-mode git-gutter:disabled-modes)
\end{verbatim}

\item Configuração
\label{sec:orgabf43ea}
\begin{verbatim}
;;; tangle/major/org.el -*- lexical-binding: t; -*-
(require 'org-src)
(add-to-list 'org-src-block-faces '("latex" (:inherit default :extend t)))

(add-hook! org-mode
    (auto-fill-mode +1)
    (setq-local real-auto-save-interval 0.2)
    (turn-off-smartparens-mode)
    (turn-on-show-smartparens-mode)
    (ws-butler-mode -1)
    (my/org-hide-properties)
    )

(set-popup-rule! "\*Org Src .+\*"
  :size 0.5)

(setq org-preview-latex-default-process 'dvisvgm)
(plist-put org-format-latex-options :scale 1.3)

(add-to-list
 '+company-backend-alist
 '(org-mode (:separate company-math-symbols-latex company-dabbrev company-ispell company-capf)))

(map! :map 'evil-org-mode-map
      :i "C-l" #'flyspell-correct-move)

(after! ox-latex
  (add-to-list 'org-latex-classes
            '("report-noparts"
                "\\documentclass{report}"
                ("\\chapter{%s}" . "\\chapter*{%s}")
                ("\\section{%s}" . "\\section*{%s}")
                ("\\subsection{%s}" . "\\subsection*{%s}")
                ("\\subsubsection{%s}" . "\\subsubsection*{%s}")
                ("\\paragraph{%s}" . "\\paragraph*{%s}")
                ("\\subparagraph{%s}" . "\\subparagraph*{%s}"))))

(map! :mode 'org-mode :g "C-S-s" #'org-latex-export-to-pdf)
\end{verbatim}

\item Hook
\label{sec:orgc17c6dd}

\begin{verbatim}
(auto-fill-mode +1)
(setq-local real-auto-save-interval 0.2)
(turn-off-smartparens-mode)
(turn-on-show-smartparens-mode)
(ws-butler-mode -1)
\end{verbatim}

\item Esconder \texttt{:properties:}
\label{sec:org25d5918}

Roubado do manual do Roam v2 (ou algo assim).

\begin{verbatim}
(defun my/org-hide-properties ()
  "Hide all org-mode headline property drawers in buffer. Could be
slow if it has a lot of overlays."
  (interactive)
  (save-excursion
    (goto-char (point-min))
    (while (re-search-forward
            "^ *:properties:\n\\( *:.+?:.*\n\\)+ *:end:\n" nil t)
      (let ((ov_this (make-overlay (match-beginning 0) (match-end 0))))
        (overlay-put ov_this 'display "")
        (overlay-put ov_this 'hidden-prop-drawer t))))
  (put 'org-toggle-properties-hide-state 'state 'hidden))

(defun my/org-show-properties ()
  "Show all org-mode property drawers hidden by org-hide-properties."
  (interactive)
  (remove-overlays (point-min) (point-max) 'hidden-prop-drawer t)
  (put 'org-toggle-properties-hide-state 'state 'shown))

(defun my/org-toggle-properties ()
  "Toggle visibility of property drawers."
  (interactive)
  (if (eq (get 'org-toggle-properties-hide-state 'state) 'hidden)
      (org-show-properties)
    (org-hide-properties)))
\end{verbatim}

\begin{verbatim}
(my/org-hide-properties)
\end{verbatim}
\end{enumerate}

\subsection{Features}
\label{sec:orgfd9efe5}
\subsubsection{@Org}
\label{sec:org7d93536}
\begin{enumerate}
\item Org-roam
\label{sec:org749418b}

\begin{verbatim}
;;; tangle/feature/org-roam.el -*- lexical-binding: t; -*-
(setq org-roam-directory "~/Lucas/notas")

(defadvice! +org-roam-reuse-windows (&rest r)
  :before #'org-roam-preview-visit
  :before #'org-roam-node-visit
  (when org-roam-buffer-current-node
    (let ((window (get-buffer-window
                    (get-file-buffer
                      (org-roam-node-file org-roam-buffer-current-node)))))
      (when window (select-window window)))))

(defadvice! doom-modeline--buffer-file-name-roam-aware-a (orig-fun)
  :around #'doom-modeline-buffer-file-name ; takes no args
  (if (s-contains-p (expand-file-name org-roam-directory) (or buffer-file-name ""))
      (replace-regexp-in-string
       "\\(?:^\\|.*/\\)\\([0-9]\\{4\\}\\)\\([0-9]\\{2\\}\\)\\([0-9]\\{2\\}\\)[0-9]*-"
       "🢔(\\1-\\2-\\3) "
       (subst-char-in-string ?_ ?  buffer-file-name))
    (funcall orig-fun)))
\end{verbatim}

\item {\bfseries\sffamily TODO} Org-ql
\label{sec:org362604f}
\begin{verbatim}
(package! org-ql)
\end{verbatim}

\item Org-sidebar\hfill{}\textsc{ARCHIVE}
\label{sec:org2cdedca}
\end{enumerate}

\subsubsection{Ispell}
\label{sec:orgaaad1bf}

\begin{verbatim}
(setq ispell-dictionary "pt_BR,en_US"
      ispell-personal-dictionary (concat doom-private-dir ".hunspell-personal"))

(unless (file-exists-p ispell-personal-dictionary)
  (write-region "" nil ispell-personal-dictionary nil 0))
\end{verbatim}

Cria um dicionário multilinguagem com português e inglês.

\begin{verbatim}
(after! ispell
  (ispell-hunspell-add-multi-dic "pt_BR,en_US")
  (ispell-set-spellchecker-params))
\end{verbatim}

\subsubsection{@Projectile}
\label{sec:org88d39a6}
\begin{verbatim}
(after! projectile
    (projectile-register-project-type 'julia '("Project.toml")
                                    :project-file "Project.toml"
                                    :test "julia -e \"using Pkg; Pkg.test()\""))
\end{verbatim}

\subsubsection{{\bfseries\sffamily TODO} Hercules}
\label{sec:org51a57c0}

O Hercules é um ``Hydra automático'': ele deixa alguns keymaps persistentes, até
que uma tecla fora do keymap seja pressionada.

\begin{verbatim}
(package! hercules)
\end{verbatim}

Exemplos:
\begin{verbatim}
(hercules-def
 :toggle-funs #'macrostep-mode
 :keymap 'macrostep-keymap)
 
(define-key <map-symbol> (kbd "<key>") #'macrostep-mode)

(hercules-def
 ;; read further to see why this works
 :toggle-funs #'org-babel-mode
 :keymap 'org-babel-map
 :transient t)
 
;; tweak binding to taste
(define-key org-mode-map (kbd "C-c C-v") #'org-babel-mode)
\end{verbatim}
\subsubsection{Treemacs}
\label{sec:orgcd5b88a}
\begin{enumerate}
\item Ignorar (esconder) algumas coisas
\label{sec:orge446025}

Roubado do tecosaur.

\begin{verbatim}
(defcustom treemacs-file-ignore-extensions
  '("aux" "ptc" "fdb_latexmk" "fls" "synctex.gz" "toc"         ;; LaTeX
    "glg"  "glo"  "gls"  "glsdefs"  "ist"  "acn"  "acr"  "alg" ;; LaTeX - glossary
    "mw"                                                       ;; LaTeX - pgfplots
    "pdfa.xmpi")                                               ;; LaTeX - pdfx
  "File extension which `treemacs-ignore-filter' will ensure are ignored"
  :safe #'string-list-p)

(defcustom treemacs-file-ignore-globs
  '("*/_minted-*"                                        ;; LaTeX
     "*/.auctex-auto" "*/_region_.log" "*/_region_.tex") ;; AucTeX
  "Globs which will are transformed to `treemacs-file-ignore-regexps'
which `treemacs-ignore-filter' will ensure are ignored"
  :safe #'string-list-p)
\end{verbatim}
\item Fringe only appears with width > 5
\label{sec:org4f3ff4b}
\begin{verbatim}
(setq doom-themes-treemacs-bitmap-indicator-width 8)
\end{verbatim}
\end{enumerate}

\subsection{Minor modes}
\label{sec:org276591d}
\subsubsection{Centaur-tabs\hfill{}\textsc{ARCHIVE}}
\label{sec:org12c1489}

\subsubsection{Elcord}
\label{sec:orgdaa57f9}

\begin{verbatim}
(package! elcord)
\end{verbatim}

\begin{verbatim}
(setq elcord-editor-icon "emacs_icon"
      elcord-display-elapsed nil
      elcord--editor-name "Emacs"
      elcord-use-major-mode-as-main-icon t)
\end{verbatim}

\subsubsection{@Evil}
\label{sec:orgcd1996c}
\begin{enumerate}
\item Variáveis
\label{sec:orgcbb7603}
\begin{verbatim}
(setq evil-shift-round nil
      evil-cross-lines t

      ;; Respeita linhas visuais
      evil-respect-visual-line-mode t

      ;; Substitui vários matches por linha no evil-ex
      evil-ex-substitute-global t)
\end{verbatim}

\item Shift
\label{sec:org94e564b}
Eu \emph{realmente} quero que o shift tenha 2 espaços na linguagem elisp.
\begin{verbatim}
;;; tangle/minor/evil.el -*- lexical-binding: t; -*-
(defadvice! ~evil-shift-width-elisp-advice (fun &rest r)
  :around #'evil-shift-width-elisp-advice
  (if (eq major-mode 'emacs-lisp-mode)
      (setq evil-shift-width 2)
    (apply fun r)))
\end{verbatim}

\item Evil-mc
\label{sec:orgde0588d}

Multicursores com o Ctrl + clique esquerdo do mouse (como no VSCode).
\begin{verbatim}
(defun evil-mc/toggle-cursor-on-click (event)
  "Add a cursor where you click, or remove a fake cursor that is
already there."
  (interactive "e")
  (mouse-minibuffer-check event)
  (require 'evil-mc)
  ;; Use event-end in case called from mouse-drag-region.
  ;; If EVENT is a click, event-end and event-start give same value.
  (let ((position (event-end event)))
    (if (not (windowp (posn-window position)))
        (error "Position not in text area of window"))
    (select-window (posn-window position))
    (let ((pt (posn-point position)))
      (if (numberp pt)
          ;; is there a fake cursor with the actual *point* right where we are?
          (unless (evil-mc-undo-cursor-at-pos pt)
            (save-excursion
              (goto-char pt)
              (evil-mc-make-cursor-here)))))))
(map! "C-<down-mouse-1>" nil)
(map! "C-<mouse-1>" #'evil-mc/toggle-cursor-on-click)
\end{verbatim}
\end{enumerate}

\subsubsection{Flyspell}
\label{sec:org7c4e731}

Vamos deixá-lo menos preguiçoso: @performance

\begin{verbatim}
(setq flyspell-lazy-idle-seconds 0.4)
\end{verbatim}

Com o \texttt{flyspell-correct}, vamos adicionar alguns atalhos:

\begin{verbatim}
(map! :ni "C-." #'flyspell-correct-move)
\end{verbatim}

\subsubsection{Iedit}
\label{sec:orgf09ad0e}

Desativa uma mensagem chata quando apertamos \texttt{M-d}

\begin{verbatim}
(setq iedit-toggle-key-default nil)
\end{verbatim}

\subsubsection{@Lisps}
\label{sec:orgccb0e39}
\begin{enumerate}
\item Parinfer
\label{sec:orga914064}

\begin{itemize}
\item \texttt{paren} Mode gives you full control of parens, while Parinfer corrects
indentation. You can still adjust indentation, but you won't be able to
indent/dedent past certain boundaries set by parens on previous lines.

\item \texttt{indent} Mode gives you full control of indentation, while Parinfer
corrects or inserts close-parens where appropriate. Specifically, it only
touches the groups of close-parens at the end of each line.

\item \texttt{smart} Mode is like Indent Mode, but it tries to preserve the structure too.
\end{itemize}

\begin{verbatim}
(setq parinfer-rust-preferred-mode "indent")
\end{verbatim}
\end{enumerate}

\subsubsection{Mamimo}
\label{sec:org4902dab}

\begin{verbatim}
(setq mamimo-greek-abbrevs-prefix "'")
(add-hook! 'mamimo-mode-hook
  (evil-tex-mode +1))
\end{verbatim}

\subsubsection{Mixed-pitch}
\label{sec:orgb5b9b2e}

\begin{verbatim}
(defface my-mixed-pitch-face
  '((t :family "Overpass" :weight light))
  "Face for `mixed-pitch-mode'")
(setq mixed-pitch-face 'my-mixed-pitch-face
      mixed-pitch-set-height nil)
\end{verbatim}

\subsubsection{@Org}
\label{sec:orge433c41}
\begin{enumerate}
\item Org-appear
\label{sec:org2109952}

Faz os elementos de markup desaparecerem automaticamente.

\begin{verbatim}
(package! org-appear)
\end{verbatim}

\begin{verbatim}
(use-package org-appear
  :hook (org-mode . org-appear-mode)
  :config
  (setq org-appear-autolinks nil))
\end{verbatim}
\end{enumerate}

\subsubsection{Vertico\hfill{}\textsc{completion}}
\label{sec:orgaf65f9f}

\begin{verbatim}
(vertico-reverse-mode +1)
(setq vertico-resize t
      vertico-count 8)
\end{verbatim}

\subsection{Carregar}
\label{sec:org3e6c555}
O trecho a seguir carrega os arquivos de configurações nos diretórios
\texttt{lisp/major} e \texttt{lisp/minor}, e faz com que cada arquivo seja carregado apenas
depois do módulo de mesmo nome ser carregado (com a função \texttt{after!}).
\begin{verbatim}
(dolist (type '(major minor features))
  (let ((folder (format "~/.doom.d/lisp/%s/" type)))
    (dolist (file (file-expand-wildcards (concat folder "*.el")))
      (let ((f (file-name-sans-extension (file-name-nondirectory file))))
        (eval `(after! ,(intern f) (load! ,f ,folder)))))))
\end{verbatim}

\begin{verbatim}
(dolist (type '(major minor features))
  (let ((folder (format "%stangle/%s/" doom-private-dir type)))
    (dolist (file (file-expand-wildcards (concat folder "*.el")))
      (let ((f (file-name-sans-extension (file-name-nondirectory file))))
        (eval `(after! ,(intern f) (load! ,f ,folder)))))))
\end{verbatim}

\section{Atalhos de teclado}
\label{sec:orga0eb357}
\subsection{{\bfseries\sffamily TODO} Implementar o atalho \texttt{C-l} no insert state}
\label{sec:orgff3161d}
\subsection{{\bfseries\sffamily TODO} Fazer o binding \texttt{SPC s I} abrir a sessão com narrowing}
\label{sec:org5db83d0}


\subsection{Comandos familiares}
\label{sec:orgf0ce997}

Porque ninguém merece tantos atalhos diferentes.

\begin{verbatim}
(map! "C-S-s" 'isearch-forward)
(map! :egni "C-s" 'save-buffer)
(map! :egni "C-/" 'evilnc-comment-or-uncomment-lines)

(map! :i "C-v" 'yank)
(map! :i "C-z" 'evil-undo)
(map! :i "C-S-z" 'evil-redo)
(map! :i "C-x" 'evil-delete)
\end{verbatim}

\subsection{Workspaces}
\label{sec:org7516ae1}
\begin{verbatim}
(map! :map 'doom-leader-workspace-map
      "}" #'+workspace/swap-right
      "{" #'+workspace/swap-left)
\end{verbatim}

\subsection{Evil ABNT\hfill{}\textsc{ARCHIVE}}
\label{sec:org7ab3ce3}

\subsection{{\bfseries\sffamily TODO} Hydras (mudar para hercules)}
\label{sec:org66ae87d}
Uma história antiga.

Não gosto do estilo do pop up
\begin{verbatim}
(setq hydra-is-helpful nil)
\end{verbatim}

Tamanho da janela
\begin{verbatim}
(defhydra window-height-hydra (evil-window-map)
  "window height"
  ("=" evil-window-increase-height "increase")
  ("-" evil-window-decrease-height "decrease"))
\end{verbatim}

\subsection{Kitty (Terminal)}
\label{sec:orga8af5e9}
\begin{verbatim}
(map! :prefix-map ("\x80" . "kitty C map")
      :map 'key-translation-map
      "/" "C-/")

(map! :prefix-map ("\x81" . "kitty C-S map")
      :map 'key-translation-map
      "z" (kbd "C-S-z"))
\end{verbatim}

\subsection{Edit leader key}
\label{sec:org101fd72}
\begin{verbatim}
(map! :leader
      :prefix ("e" . "edit")
      :desc "New snipet" "s" #'+snippets/new
      :desc "New alias" "a" #'+snippets/new-alias)

(map! :i "C-M-x" ctl-x-map)
\end{verbatim}

\subsection{Centralizar ao procurar com o evil}
\label{sec:org7e7e23a}
\begin{verbatim}
;; (evil-define-motion search-previous-and-recenter (count)
;;   :jump t
;;   :type exclusive
;;   (evil-ex-search-previous count)
;;   (call-interactively #'evil-scroll-line-to-center))

;; (map! :n [remap evil-ex-search-previous] #'search-previous-and-recenter)
\end{verbatim}

\subsection{Outros}
\label{sec:orgd66ba59}

\begin{verbatim}
(map! "M-j" 'drag-stuff-down
      "M-k" 'drag-stuff-up)

(map! :map lean-mode-map "M-." 'lean-find-definition)

(map! :map TeX-mode-map "C-S-s" 'TeX-command-run-all)
\end{verbatim}
\end{document}
